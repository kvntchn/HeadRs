%---------------------------------------------------------------------------
% Tables & figures
%---------------------------------------------------------------------------

% For framing
\usepackage{framed}

% wrapped figures
\usepackage{wrapfig}

% Witin cell wrapping
% https://tex.stackexchange.com/questions/2441/how-to-add-a-forced-line-break-inside-a-table-cell
\usepackage{makecell}
\renewcommand\theadset{\def\arraystretch{.85}}%
\renewcommand\cellset{\def\arraystretch{.85}}%
\renewcommand\theadalign{bc}
\renewcommand\theadfont{\bfseries}
\renewcommand\theadgape{\Gape[4pt]}
\renewcommand\cellgape{\Gape[4pt]}
% Usage
% \thead{Two line \\ head}
% \makecell{Some really \\ longer text}

\ifluatex
% requires lualatex
\usepackage{pgf}
% https://tex.stackexchange.com/questions/34939/axis-with-trigonometric-labels-in-pgfplots
\usepackage{pgfplots}
\fi
\pgfplotsset{compat=1.17}
\usepackage{tikz}% Probability trees and flow charts and the like
\usetikzlibrary{
graphs,
arrows,
automata,
%trees,
%shapes,
decorations,
positioning
%backgrounds,
%petri
}
% https://tex.stackexchange.com/questions/451786/how-do-i-put-a-circle-around-a-symbol
\usepackage{xargs}
\newcommandx\circled[3][1=0.75, 2=0.5pt]{
\begin{tikzpicture}[baseline=(char.base), scale = {#1}, every node/.style={scale = {#1}}]
  \node[shape = circle, draw, inner sep = {#2}] (char) {{#3}};
\end{tikzpicture}
	}
\ifluatex
% requires lualatex
\usetikzlibrary{graphdrawing}
\fi

%---------------------------------------------------------------------------
% Math supplement
%---------------------------------------------------------------------------
% \usepackage{hologo}
\ifnum 0\ifxetex 1\fi\ifluatex 1\fi=0 % if pdftex
% TIPA (not compatible with fontspec)
\usepackage{tipx}
\fi
\usepackage{amsmath,amsfonts,amsthm,amssymb}
\usepackage{mathtools} %\mathtoolsset{showonlyrefs} % showonlyrefs removes equation numbers
\usepackage{bm} % greek no good in lualatex
% \usepackage{physics}
% \usepackage{marvosym,epsdice}
% \usepackage{MnSymbol} % good-looking extended math symbols
% \usepackage{lilyglyphs} % Extended music symbols (needs XeLaTex of LuaTex)
\usepackage{dsfont} % for cool indicator
\usepackage{textcomp} % because unicode and hyperref may have issues

% https://tex.stackexchange.com/questions/174118/not-independent-sign-in-latex
\makeatletter
\newcommand*{\indep}{%
  \mathbin{%
    \mathpalette{\@indep}{}%
  }%
}
\newcommand*{\nindep}{%
  \mathbin{%                   % The final symbol is a binary math operator
    \mathpalette{\@indep}{\not}% \mathpalette helps for the adaptation
                               % of the symbol to the different math styles.
  }%
}
\newcommand*{\@indep}[2]{%
  % #1: math style
  % #2: empty or \not
  \sbox0{$#1\perp\m@th$}%        box 0 contains \perp symbol
  \sbox2{$#1=$}%                 box 2 for the height of =
  \sbox4{$#1\vcenter{}$}%        box 4 for the height of the math axis
  \rlap{\copy0}%                 first \perp
  \dimen@=\dimexpr\ht2-\ht4-.2pt\relax
      % The equals symbol is centered around the math axis.
      % The following equations are used to calculate the
      % right shift of the second \perp:
      % [1] ht(equals) - ht(math_axis) = line_width + 0.5 gap
      % [2] right_shift(second_perp) = line_width + gap
      % The line width is approximated by the default line width of 0.4pt
  \kern\dimen@
	\ifx\\#2\\%
  \else
    \hbox to \wd2{\hss$#1#2\m@th$\hss}%
    \kern-\wd2 %
  \fi
  \kern\dimen@
  \copy0 %                       second \perp
}
\makeatother

% % limits underneath
% \DeclareMathOperator*{\argminA}{arg\,min} % Jan Hlavacek
% \DeclareMathOperator*{\argminB}{argmin}   % Jan Hlavacek
% \DeclareMathOperator*{\argminC}{\arg\min}   % rbp
% \newcommand{\argminD}{\arg\!\min} % AlfC
% \newcommand{\argminE}{\mathop{\mathrm{argmin}}}          % ASdeL
% \newcommand{\argminF}{\mathop{\mathrm{argmin}}\limits}   % ASdeL

\newcommand{\Prob}[2][]{ \mathbb{P}_{#1} \left( {#2} \right)} % make a function for bb bold P with curly braces

\newcommand{\E}[2][]{ \mathbb{E}_{#1}\left[ {#2} \right]} % make a function for expectation

\newcommand{\Var}[2][]{ \mathrm{Var}_{#1}\left( {#2} \right)} % function for Var

\newcommand{\Cov}[2][]{ \mathrm{Cov}_{#1}\left( {#2} \right)} % function for Covar

\newcommand{\lik}[2][]{ \mathrm{lik}_{#1}\left( {#2} \right)} % function for lik

\newcommand{\Ind}[2][]{\mathds{1}_{#1}\left[ {#2} \right]} % function for Indicator

% Stacked lines (within line stacking)
\usepackage{stackengine}
\def\stackalignment{l}
% Usage
% \addstackgap[2pt]{\stackanchor[2.5pt]{Temper force}{with humanity}}

%---------------------------------------------------------------------------
% Code chunks and color
%---------------------------------------------------------------------------
% \usepackage{cprotect}
\usepackage{listings} % for code
\usepackage{tcolorbox}
\usepackage{xparse} % need for NewDocumentCommand
\NewDocumentCommand{\framecolorbox}{oommm}
 {% #1 = width (optional)
  % #2 = inner alignment (optional)
  % #3 = frame color
  % #4 = background color
  % #5 = text
  \IfValueTF{#1}
   {\IfValueTF{#2}
    {\fcolorbox{#3}{#4}{\makebox[#1][#2]{#5}}}
    {\fcolorbox{#3}{#4}{\makebox[#1]{#5}}}%
   }
   {\fcolorbox{#3}{#4}{#5}}%
 }

% \usepackage{color}
%\definecolor{codegreen}{rgb}{0,0.6,0}
%\definecolor{codegray}{rgb}{0.5,0.5,0.5}
%\definecolor{codepurple}{rgb}{0.58,0,0.82}
%\definecolor{backcolour}{rgb}{0.95,0.95,0.92}
%
%\lstdefinestyle{mystyle}{
%backgroundcolor=\color{backcolour},
%commentstyle=\color{codegreen},
%keywordstyle=\color{magenta},
%numberstyle=\tiny\color{codegray},
%stringstyle=\color{codepurple},
%basicstyle=\footnotesize,
%breakatwhitespace=false,
%breaklines=true,
%captionpos=b, keepspaces=true, numbers=left, numbersep=5pt, showspaces=false,
%showstringspaces=false, showtabs=false, tabsize=2 }
%
%\lstset{style=mystyle}


\usepackage{hyperref}
\hypersetup{unicode=true,
            pdfauthor={Kevin Chen},
            pdfborder={0 0 0},
            breaklinks=true}
% \usepackage{url}

% Code chunk captions and references
% \usepackage{floatrow} %interferes with float package
% \floatsetup[figure]{capposition=top}
% \floatsetup[table]{capposition=top}
% \DeclareNewFloatType{chunk}{placement=H, fileext=chk, name=}
% \captionsetup{options=chunk}
% \renewcommand{\thechunk}{Chunk~\arabic{chunk}}
% \makeatletter
% \@addtoreset{chunk}{section}
% \makeatother

% Usage in R
% library(knitr)
% oldSource <- knit_hooks$get("source")
% knit_hooks$set(source = function(x, options) {
%   x <- oldSource(x, options)
%   x <- ifelse(!is.null(options$ref), paste0("\\label{", options$ref,"}", x), x)
%   ifelse(!is.null(options$codecap), paste0("\\captionof{chunk}{", options$codecap,"}", x), x)
% })

%---------------------------------------------------------------------------
% Including R chunks (LaTex)
%---------------------------------------------------------------------------

\lstloadlanguages{R} % Load R syntax for listings, for a list of other languages supported see: ftp://ftp.tex.ac.uk/tex-archive/macros/latex/contrib/listings/listings.pdf
\definecolor{gray}{rgb}{0.5,0.5,0.5}
\definecolor{green}{rgb}{0,0.6,0}
\definecolor{mauve}{rgb}{0.58,0,0.82}
\definecolor{maroon}{rgb}{0.3,0.1,0.9}
\lstset{
language=R,
basicstyle=\scriptsize\ttfamily,
commentstyle=\ttfamily\color{gray},
numbers=left,
numberstyle=\ttfamily\color{gray}\footnotesize,
stepnumber=1,
numbersep=5pt,
backgroundcolor=\color{white},
showspaces=false,
showstringspaces=false,
showtabs=false,
frame=single,
tabsize=2,
captionpos=b,
breaklines=true,
breakatwhitespace=false,
% title=\lstname,
escapeinside={},
keywordstyle=\ttfamily\color{green},
stringstyle=\ttfamily\color{mauve},
identifierstyle=\color{blue},
morekeywords={}
}

% Usage
	% For use in LaTex or Sweave:
	% \lstinputlisting[language=R]{Script.R}

%---------------------------------------------------------------------------
% Environment commands for Markdown
%---------------------------------------------------------------------------
\usepackage{pdflscape}
\newcommand{\blandscape}{\begin{landscape}}
\newcommand{\elandscape}{\end{landscape}}

\newcommand{\bcenter}{\begin{center}}
\newcommand{\ecenter}{\end{center}}

\newcommand{\bsinglespace}{\begin{singlespace}}
\newcommand{\esinglespace}{\end{singlespace}}

\newcommand\startsupplement{%
    \makeatletter
       \setcounter{table}{0}
       \renewcommand{\thetable}{S\arabic{table}}
       \setcounter{figure}{0}
       \renewcommand{\thefigure}{S\arabic{figure}}
    \makeatother}
