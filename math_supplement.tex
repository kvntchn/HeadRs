%---------------------------------------------------------------------------
% Math supplement
%---------------------------------------------------------------------------

% \usepackage{hologo}
\ifnum 0\ifxetex 1\fi\ifluatex 1\fi=0 % if pdftex
  % TIPA (not compatible with fontspec)
  \usepackage{tipx}
\fi
\usepackage{amsmath,amsfonts,amsthm,amssymb}
\usepackage{mathtools} %\mathtoolsset{showonlyrefs} % showonlyrefs removes equation numbers
\usepackage{bm} % greek no good in lualatex
% \usepackage{physics}
% \usepackage{marvosym,epsdice}
% \usepackage{MnSymbol} % good-looking extended math symbols
% \usepackage{lilyglyphs} % Extended music symbols (needs XeLaTex of LuaTex)
\usepackage{dsfont} % for cool indicator
\usepackage{textcomp} % because unicode and hyperref may have issues
\usepackage{upgreek} % upright greek

% https://tex.stackexchange.com/questions/174118/not-independent-sign-in-latex
\makeatletter
\newcommand*{\indep}{%
  \mathbin{%
    \mathpalette{\@indep}{}%
  }%
}
\newcommand*{\nindep}{%
  \mathbin{%                   % The final symbol is a binary math operator
    %\mathpalette{\@indep}{\not}% \mathpalette helps for the adaptation
    \mathpalette{\@indep}{/}%
    % of the symbol to the different math styles.
  }%
}
\newcommand*{\@indep}[2]{%
  % #1: math style
  % #2: empty or \not
  \sbox0{$#1\perp\m@th$}%        box 0 contains \perp symbol
  \sbox2{$#1=$}%                 box 2 for the height of =
  \sbox4{$#1\vcenter{}$}%        box 4 for the height of the math axis
  \rlap{\copy0}%                 first \perp
  \dimen@=\dimexpr\ht2-\ht4-.15pt\relax
  % The equals symbol is centered around the math axis.
  % The following equations are used to calculate the
  % right shift of the second \perp:
  % [1] ht(equals) - ht(math_axis) = line_width + 0.5 gap
  % [2] right_shift(second_perp) = line_width + gap
  % The line width is approximated by the default line width of 0.4pt
  \kern\dimen@
  \ifx\\#2\\%
  \else
    \hbox to \wd2{\hss$#1#2\m@th$\hss}%
    \kern-\wd2 %
  \fi
  \kern\dimen@
  \copy0 %                       second \perp
}
\makeatother

\newcommand{\ind}{\perp\mkern-10mu\perp}

% % limits underneath
% \DeclareMathOperator*{\argminA}{arg\,min} % Jan Hlavacek
% \DeclareMathOperator*{\argminB}{argmin}   % Jan Hlavacek
\DeclareMathOperator*{\argmin}{\arg\min}   % rbp
% \newcommand{\argminD}{\arg\!\min} % AlfC
% \newcommand{\argminE}{\mathop{\mbox{argmin}}}          % ASdeL
% \newcommand{\argminF}{\mathop{\mbox{argmin}}\limits}   % ASdeL
\DeclareMathOperator*{\argmax}{\arg\max}   % rbp

% \newcommand{\myqed}{\flushright\rule{.36em}{1.5ex}}
\newcommand{\Prob}[2][]{ \mathbb{P}_{#1} \left( {#2} \right)} % make a function for bb bold P with curly braces

\newcommand{\E}[2][]{ \mathbb{E}_{#1}\left[ {#2} \right]} % make a function for expectation

\newcommand{\Var}[2][]{\mbox{var}_{#1}\left( {#2} \right)} % function for Var

\newcommand{\Cov}[2][]{\mbox{cov}_{#1}\left( {#2} \right)} % function for Covar

\newcommand{\Cor}[2][]{\mbox{cor}_{#1}\left( {#2} \right)} % function for Cor

\newcommand{\lik}[2][]{\mbox{lik}_{#1}\left( {#2} \right)} % function for lik

\newcommand{\bias}[2][]{\mbox{bias}_{#1}\left( {#2} \right)} % function for bias

\newcommand{\tr}[2][]{\mbox{tr}_{#1}\left( {#2} \right)} % function for trace

\newcommand{\rss}[2][]{\textsc{rss}_{#1}\left( {#2} \right)} % function for RSS

\newcommand{\tss}[2][]{\textsc{tss}_{#1}\left( {#2} \right)} % function for TSS

\newcommand{\diag}[2][]{\mbox{diag}_{#1}\left( {#2} \right)} % function for diag

\newcommand{\Ind}[2][]{\mathds{1}_{#1}\left[ {#2} \right]} % function for Indicator

% new \oset macro:
\makeatletter
\newcommand{\oset}[3][0.1ex]{%
  \mathrel{\mathop{#3}\limits^{
      \vbox to#1{\kern-2\ex@
        \hbox{$\scriptstyle#2$}\vss}}}}
\makeatother

% Stacked lines (within line stacking)
\usepackage{stackengine}
\def\stackalignment{l}
% Usage
% \addstackgap[2pt]{\stackanchor[2.5pt]{Temper force}{with humanity}}
